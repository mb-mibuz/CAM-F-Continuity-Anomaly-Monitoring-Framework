\doublespacing % Do not change - required

\chapter{Reflections}
\label{chLast}

%%%%%%%%%%%%%%%%%%%%%%%%%%%%%%%%%%%%%%%
% IMPORTANT
\begin{spacing}{1} %THESE FOUR
\minitoc % LINES MUST APPEAR IN
\end{spacing} % EVERY
\thesisspacing % CHAPTER
% COPY THEM IN ANY NEW CHAPTER
%%%%%%%%%%%%%%%%%%%%%%%%%%%%%%%%%%%%%%%

{\color{red}
\begin{itemize}
\item Always keep this as the last chapter of your thesis (before the conclusions).
\item Each section below is developed to demonstrate Learning Outcomes as formulated by IET.
\item Note that if you think that some of these sections are not relevant, you can say so. For example: \textit{This project is purely theoretical as it is dedicated to proving that the number of prime numbers is infinite. As such this project has no ethical implications.}
\item The above approach is preferable to removing the sections altogether because some of the sections may be compulsory for your stream.
If you want to erase these sections, please first check the requirements of your stream.
\end{itemize}}


\section{Legal and Ethical matters}

This section should contain a concise discussion of relevant legal and ethical aspects related to the project.

\textbf{Legal considerations.}
This part should demonstrate awareness of legal aspects relevant to engineering practice and project development. It may include topics such as intellectual property rights (e.g., ownership of code or designs), regulatory compliance, and data protection laws.

\smallskip

\textbf{Ethical considerations.}
Students should identify any ethical issues raised by the project and explain how they addressed them. This may include issues related to data use and safety. The discussion should show the student's ability to make reasoned ethical decisions within the engineering context.





\section{Environmental and Social impact}


This section should demonstrate how the sustainability and environmental and societal impact of the proposed solutions is assessed and potential adverse impacts (if any) are identified, together with suggestions for possible ways to mitigate them. 
We recognize that some projects may only have a rather theoretical flavour, so that any societal impact could be hard to anticipate. Nevertheless, an effort should be made, while identifying potential application scenarios, to also comment on their sustainability and potential undesired adverse impacts.
%This will demonstrate the following Learning Outcome as required by IET: ``Evaluate the environmental and societal impact of solutions to complex problems (to include the entire life-cycle of a product or process) and minimise adverse impacts''. %There is no prescribed length for this section, but its presence is a mandatory requirement in your final submission file.

%You can move this section somewhere else in the report if you wish as long as the report contains a dedicated section, rather than having such considerations spread across the manuscript.


\section{Equality, Diversity, and Inclusion}

This section should reflect an understanding of the importance of equality, diversity, and inclusion (EDI) within engineering practice.


Students should explain how EDI principles have been considered in the project, whether in the design process, user accessibility, team collaboration, or stakeholder engagement. This may include considerations such as accessibility of technologies, bias mitigation in data or algorithms, cultural awareness, or inclusive communication. The section should demonstrate an appreciation of the responsibilities and benefits associated with promoting an inclusive engineering environment.

\section{Quality Management Systems}

This section should provide a brief discussion of how quality management principles have been applied or considered during the project.

This may involve reference to established frameworks, the use of testing protocols, validation procedures, or version control systems used during the project. The focus should be on how quality was monitored, evaluated, and improved throughout the work.